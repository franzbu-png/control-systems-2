Mit numerischen Verfahren kann man für fixe Regelstrukturen (z.B: PI) ein Optimierungsproblem lösen, um die Parameter $k_p,\, T_i$ zu bestimmen. Dazu stellt man eine Kostenfunktion $J(k_p,T_i)$ auf, welche man durch die Wahl der optimalen Pramater $k_p^\star$ und $T_i^\star$ minimieren will.

\textbf{Beispiele für Gütekriterien:}
\begin{itemize}
    \item kumulativer Fehler über den quadrieten Fehler:
        \begin{equation*}
            g_1 = \int_0^\infty e^2(t)dt
        \end{equation*}
    
    \item Der maximale Überchuss von $y(t)$:
        \begin{equation*}
            g_2 = \max_{t\in[0,\infty)}(y(t)-1)
        \end{equation*}
    
    \item Robustheit über Minimum return difference $\mu$\textsubscript{min}
        \begin{equation*}
            g_3 = 1 -\mu_{\textnormal{min}} = 1 - \max_{\omega\in[0,\infty)}|1 + L(\jw)|
        \end{equation*}
\end{itemize}

Die Kostenfunktion ist daher $J(k_p,T_i) = \kappa_1\cdot g_1 + \kappa_2\cdot g_2 + \kappa_3\cdot g_3$. Das resultierende Optimierungsproblem 
\begin{equation*}
    \min_{k_p,T_i}J(k_p,T_i)    
\end{equation*}
kann in MATLAB mit \texttt{fminsearch} gelöst werden, wobei $\kappa_{(\cdot)}$ ein Mass für die Wichtigkeit einer Grösse $g_{(\cdot)}$ sind. Je Grösser ein $\kappa_{(\cdot)}$ ist, desto mehr wird das dazugehörige $g_{(\cdot)}$ bestraft. Dadurch werden die anderen Kriterien, relativ zum höher gewichtetn Kriterium, weniger gewichtet.


