\subsection{Singulärwerte}
    Im folgenden werden Eigenschaften zur linearen Abbildung\[y= M \cdot u,\ u \in \mathbb{C}^m,\ y\in\mathbb{C}^p,\ M\in\mathbb{C}^{p\times m}\]
    beschrieben. Insbesondere erfüllt der Ausgang $y$ folgende wichtige Eigenschaft:
    \begin{equation*}
    \colorboxed{red}{\sigma_{\textnormal{min}}(M)\leq\frac{||y||}{||u||}\leq\sigma_{\textnormal{max}}(M),}
    \end{equation*}
    wobei $||.||$ die euklische Norm ist und 
    \begin{equation*}
        \colorboxed{red}{\sigma_i(M)=\sqrt{\lambda_i(\overline{M}^T\cdot M)}>0,}
    \end{equation*}

    Die Singulärwerte der Matrix $M$ sind.

    $\overline{M}^T$ ist die Transponierte der Komplex-Konjugierten von M. Die Singulärwertzerlegung (SVD) der Matrix $M$ lautet:
    \[M = U\cdot\Sigma \cdot V^T,\]
    Wobei $\Sigma$ die \textbf{Singulärwerte $\sigma_i$ auf der Hauptdiagonale} hat.
    $U$ und $V^T$ sind unitäre (längenerhaltende) Transformationsmatrizen:
    \[U\cdot U^T = V\cdot V^T = I\]
    note: Die Anzahl Singulärwerte eines Systems mit $q$-Eingängen und $P$-Ausgängen ist: $min(p,q)$.
    
    \subsubsection{Vorgehen:}
        \begin{enumerate}
            \item \textbf{Singulärwerte bestimmen:}
                \[\sigma_i =\sqrt{\lambda_i(\overline{M}^TM)}=\sqrt{\lambda_i(M\overline{M}^T)}\]
            \item \textbf{$\Sigma \in                    \mathbb{R}^{P\times m}$ konstruieren}
                \[\Sigma = \left[\begin{array}{c c c | c}
                \sigma_1 & 0 & \dots  & 0 \\
                0&\ddots & 0 & \vdots \\
                \vdots & 0 & \sigma_r & 0 \\ \hline
                0 & \dots & 0 & 0              \end{array}\right]\ \sigma_1 \geq \sigma_2 \geq \dots \geq \sigma_r \geq 0 \] 
            \item $v_i$: Normalisierte Eigenvektoren     von $M^TM \Rightarrow$                   input-Richtungen
            \item $u_i$: Normalisierete Eigenvektoren von $MM^T
                \Rightarrow $ output-Richtungen
        \end{enumerate}
note: Falls $\sigma_i \neq 0$ geht auch: $u_i = \frac{1}{\sigma_i}Mv_i$
