\subsection{SISO - Recap}
    Ein lineares, asymptotisch stabiles System $P(s)$, welches mit einem harmonischen Eingang
    \begin{equation*}
        u(t) = \cos\left(\omega t\right) \cdot h(t)
    \end{equation*}
    bei einer fixen Frequenz $\omega$ angeregt wird, produziert im eingeschwungenen Zustand ein harmonisches Signal bei derselben Frequenz:
    \begin{equation*}
        y_\infty(t) = |P(\jw)|\cdot\cos\left(\omega t + \angle P(\jw)\right)
    \end{equation*}
    Dabei ist die Systemantwort um $\varphi(\omega) = \angle P(\jw)$ phasenverschoben und um $ m(\omega) = |P(\jw)|$ skaliert. Um die Frequenzantwort übersichtlich darzustellen, können $|P(\jw)|$ und $\angle P(\jw)$ im Nyquist- oder Bode-Plot dargestellt werden.