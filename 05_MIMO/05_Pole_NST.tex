\subsection{Pole und Nullstellen} 
    % Pole und Nullstellen sind im MIMO Fall etwas komplizierter.
\subsubsection{Definition Matrix Minoren}
        Minoren einer Matrix sind die Determinanten aller quadratischen Submatrizen. Die Submatrizen werden durch Streichen einzelner Zeilen und Spalten der Matrix gebildet.
        
        \begin{equation*}
            A = 
            \begin{bmatrix} 
                A_{11} & A_{12} & A_{13}\\
                A_{21} & A_{22} & A_{23}\\ 
            \end{bmatrix}
        \end{equation*}
        Die Minoren der Matrix $A$ sind:\\ $A_{11},\,A_{12},\,A_{13},\,A_{21},\,A_{22},\,A_{23},\,A_{31},\,A_{32},\,A_{33}, \\ \,A_{11}A_{22}-A_{21}A_{12},\,A_{11}A_{23}-A_{21}A_{13},\,A_{12}A_{23}-A_{22}A_{13}$.

    \subsubsection{Polstellen von MIMO Systemen}
        Die Pole von $P(s)$ sind die Nullstellen des kleinsten gemeinsamen Vielfachen (kgV) der Nennerpolynome aller Minoren von $P(s)$

    \subsubsection{Nullstellen von MIMO Systemen}
        Die Nullstellen von $P(s)$ sind die Nullstelle des grössten gemeinsamen Teilers (ggT) der Zähler der Minoren höchster Ordnung von $P(s)$ nach der Normalisierung, bei der alle Pole von $P(s)$ im Nenner stehen.
        
        Note: Nullstellen $s=\zeta_i$ sind nicht triviale Frequenzen, bei denen für ein spezifisches Eingangssignal $u(t)$ und spezifische Anfangsbedingungen $x(0)$ gilt: $y(t) = 0 \,\forall t$

        D.h. die folgende Laplace Transformation erfüllt
        \begin{align*}
            (sI_{n\times n} -A)\cdot x - B \cdot u &= 0\\
            C\cdot x + D \cdot u &= 0
        \end{align*}
        wobei nur eine nichttriviale Lösung besteht, falls die folgende Matrix singulär ist: 
        \begin{equation*}
            \det\begin{bmatrix}
                (sI_{n\times n}- A) & -B \\
                C& D
            \end{bmatrix} = 0
        \end{equation*}

