    \subsubsection{Bsp}
        Wir wollen ein MIMO System mit einen P-Regler regeln. Dabei vernachlässigen wir die Kreuzkopplungen und gehen davon aus, dass $u_1 \rightarrow y_1$ und $u_2 \rightarrow y_2$ regelt.
        \begin{equation*}
            P(s) =
            \begin{bmatrix}
            \frac{2}{s+1}  &    \frac{3}{s+2}\\
            \frac{1}{s+1}  &    \frac{1}{s+1}
            \end{bmatrix}
            \qquad
            C(s) = 
            \begin{bmatrix}
            K_1 & 0\\
            0   & K_2
            \end{bmatrix}
        \end{equation*}
        \textit{Q: Wie gross können wir $K_i$ wählen, ohne die Stabilität zu gefährden?}
        
        A: 
        \begin{gather*}
            T_1(s) = \frac{K_1\frac{2}{s+1}}{1 + K_1\frac{2}{s+1}} = \frac{2K_1}{s +(2K_1 + 1)},\\
            T_2(s) = \frac{K_2\frac{1}{s+1}}{1 + K_2\frac{1}{s+1}} = \frac{K_2}{s + (K_2 + 1)}\\
            \Rightarrow 2K_1 + 1 > 0, \qquad K_2 + 1 > 0\\
            \Rightarrow K_1 > -\frac{1}{2},\, K_2 > -1,\, \textnormal{bzw:}\, K_i > 0
        \end{gather*}
        
    \subsubsection{Bsp}
        Bestimme die Pol- und Nullstellen von
        \begin{equation*}
            P(s) = 
            \begin{bmatrix}
            \frac{1}{s+1}   &   0   &   \frac{s-1}{(s+1)(s+2)}\\
            \frac{-1}{s-1}  &   \frac{1}{s+2} & \frac{1}{s+2}
            \end{bmatrix}
        \end{equation*}
        bestimmen.
        
        Für die Pole bestimmen wir zuerst die Minoren:
        \begin{gather*}
            \frac{1}{s+1},\qquad 0,\qquad \frac{s-1}{(s+1)(s+2)}\\
            \frac{-1}{s-1},\qquad \frac{1}{s+2},\qquad \frac{1}{s+2}\\
            \frac{1}{(s+1)(s+2)},\qquad \frac{2}{(s+1)(s+2)},\qquad \frac{(-1)(s-1)}{(s+1)(s+2)^2}
        \end{gather*}
        KGV\textsubscript{Nenner} ist daher:
        \begin{gather*}
            (s+1)(s+2)^2(s-1)\\
            \pi_1 = -1,\quad \pi_{2,3} = -2,\quad \pi_4 = 1
        \end{gather*}
        
        Um die (Übertragungs-)Nullstellen des Systems zu bestimmen erweitern wir die Minoren höchster Ordnung (Von den ``grössten" Determinanten):
        \begin{equation*}
            \frac{\textcolor{red}{(s+2)(s-1)}}{(s+1)(s+2)\textcolor{red}{^2(s-1)}},\quad \frac{2\textcolor{red}{(s+2)(s-1)}}{(s+1)(s+2)\textcolor{red}{^2(s-1)}},\quad \frac{(-1)(s-1)\textcolor{red}{^2}}{(s+1)(s+2)^2\textcolor{red}{(s-1)}}
        \end{equation*}
        
        GGT\textsubscript{Zähler} ist somit:
        \begin{equation*}
            (s-1) \Rightarrow \zeta_1 = 1
        \end{equation*}
        
        \textit{Q: Was ist die minimale Ordnung einer internen Beschreibung?}
        
        A: Es wurden 4 Pole gefunden, somit besitzt eine interne Beschreibung des Systems mindestens 4. Ordnung. \textbf{Bemerkung:} Ein Mehrgrössensystem kann Pole und NST bei den gleichen Frequenzen haben, ohne dass sich diese kürzen, da es darauf ankommt, ob die dieselbe Frequenz \textit{und} Richtung haben.
        