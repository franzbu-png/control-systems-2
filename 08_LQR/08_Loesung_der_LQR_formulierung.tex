\subsection{Lösung der LQR-Formulierung}
    Die Lösung der LQR-Formulierung ist eine lineare Zustandsrückführung und lautet  
    \[
    \colorboxed{red}{
    u^*(t) = -K\cdot x(t),\ \textnormal{wobei}\ K = R^{-1}\cdot B^T\cdot\Phi
    }
    \]
    
    Note: Die Matrix $K$ ist statisch, sie muss für gegebene $\{A,B,Q,R\}$ nur einmal berechnet werden.

    \subsubsection{algebraische Riccati Gleichung}
        Dabei ist $\Phi$ die einzige positive Lösung der algebraischen Riccati Gleichung 
        \[
        \colorboxed{red}{
        \Phi\cdot B \cdot R^{-1}\cdot B^T \cdot \Phi-\Phi \cdot A - A^T \cdot \Phi - Q = 0
        }
        \]
        Wählt man \[Q = \overline{C}^T\cdot \overline{C},\quad \overline{C}\in\mathbb{R}^{p\times n}\ \textnormal{wobei}\ p = \operatorname{rank}(Q),\]
        dann ist $\Phi$ garantiert positiv definit, falls $\{A,B\}$ \textbf{steuerbar} und $\{A,\Bar{C}\}$ \textbf{beobachtbar} sind. Diese Bedingungen sind hinreichen aber nicht notwendig.
        
        Note:\begin{enumerate}
                \item  Die Matrix $\overline{C}$ hat nichts mit der Matrix C zu tun. Falls jedoch $\overline{C} = C$ gewählt wird, wird $||y(t)||_2$ in der Kostenfunktion berücksichtigt.%, da:
                % \[ x^T\cdot Q \cdot x = x^T \cdot\overline{C}^T\cdot \overline{C} \cdot x = x^T \cdot C^T \cdot C \cdot x = y^T \cdot y = \|y\|_2.\]
                \item Die Dynamik des geregelten Systems lautet $\dot x = (A-BK)x$. Die Matrix $A-BK$ ist garantiert Hurwitz. D.h. der geschlossene Regelkreis ist asymptotisch stabil.
                \item Der open loop gain lautet $L_{LQR}(s) = K\cdot (sI-A)^{-1}\cdot B$ %``Der open loop gain betrachtet den offenen Pfad vom roten zum blauen Punkt, wobei beim blauen Punkt aufgeschnitten wir" \textbf{Bild hinzfügen?}
            \end{enumerate}
    